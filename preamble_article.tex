\usepackage[round,authoryear,sort]{natbib}
%\usepackage[round,authoryear,sort]{kznnat}
%\usepackage[danish]{babel}
\usepackage{epsfig,longtable}
\usepackage{rotating}
\usepackage{amsfonts}
\usepackage{amssymb} 
\usepackage{verbatim} 
\usepackage{fancyhdr}
%\usepackage{../latex_stuff/a+a}
\pagestyle{fancy} 

\renewcommand{\textwidth}{14.0cm}
\setlength{\headwidth}{14.0cm}
\renewcommand{\oddsidemargin}{1.5cm}
\renewcommand{\evensidemargin}{1.0cm}

%\renewcommand{\chaptermark}[1]{\markboth{#1}{}}
%\renewcommand{\sectionmark}[1]{\markright{\thesection\ #1}}


\fancyhf{}
\fancyhead[LE,RO]{\bfseries\thepage}
\fancyhead[LO]{\bfseries\rightmark}
\fancyhead[RE]{\bfseries\leftmark}
\renewcommand{\headrulewidth}{0.5pt}
\renewcommand{\footrulewidth}{0pt}
\addtolength{\headheight}{0.5pt}


%\setlength{\oddsidemargin}{1.5cm}
%\setlength{\evensidemargin}{0.8cm}
%\setlength{\oddsidemargin}{0.0cm}
%\setlength{\evensidemargin}{0.0cm}
%\setlength{\textwidth}{14.5cm}

\fancypagestyle{plain}{%
   \fancyhead{}
   \renewcommand{\headrulewidth}{0pt}
}

% Change definitions for floats
% min fraction of text on a page
\renewcommand{\textfraction}{0.0}
\renewcommand{\floatpagefraction}{0.7}



%\bibliographystyle{kznplnat}
%\bibliographystyle{mrk_plainnat}
%\bibliographystyle{plainnat}
%\bibliographystyle{../latex_stuff/a+a}
%\bibliographystyle{../latex_stuff/astron/astron}
\bibliographystyle{../latex_stuff/myastron/myastron}


%\addtolength{\marginparwidth}{-70pt}
%\addtolength{\textwidth}{70pt}
\addtolength{\textheight}{0.5in}
\addtolength{\voffset}{-0.5in}
%\addtolength{\hoffset}{-0.5in}
\newlength{\tabwidth}

\renewcommand{\textfraction}{0}

\newcommand{\tabtitle}[1]{{\small\textit{#1}}}
\newcommand{\tabelcapt}[2]{\caption{\label{#1}\tabtitle{#2}}}
\newcommand{\tablecapt}[2]{\caption{\label{#1}\tabtitle{#2}}}

\newcommand{\figurcapt}[2]{\caption{\label{#1}\small #2}}

\newcommand{\figpsr}[4]{
\begin{figure}[!htbp]
\begin{center}
\begin{turn}{90}
\epsfig{file=#1.ps, width=#3cm}
\end{turn}
\end{center}
\vspace{-1cm}
\figurcapt{#1}{#4}
\vspace{+1cm}
\end{figure}
}
\newcommand{\figpsrback}[4]{
\begin{figure}[!htbp]
\begin{center}
\begin{turn}{-90}
\epsfig{file=#1.ps, width=#3cm}
\end{turn}
\end{center}
%\vspace{-1cm}
\figurcapt{#1}{#4}
%\vspace{+1cm}
\end{figure}
}
%\makebox[14cm]{\framebox[14cm]{\rule{0pt}{9cm}}}

\newcommand{\figps}[4]{
\begin{figure}[!htbp]
\begin{center}
\epsfig{file=#1.ps, height=#2cm, width=#3cm}
\end{center}
%\vspace{-.5cm}
\figurcapt{#1}{#4}
%\vspace{+.5cm}
\end{figure}
}

\newcommand{\figpsauto}[3]{
\begin{figure}[!htbp]
\begin{center}
\epsfig{file=#1.ps, width=#2cm}
\end{center}
%\vspace{-.5cm}
\figurcapt{#1}{#3}
%\vspace{+.5cm}
\end{figure}
}

\newcommand{\figpsautoha}[3]{
\begin{figure}[!h]
\begin{center}
\epsfig{file=#1.ps, width=#2cm}
\end{center}
\vspace{-1.0cm}
\figurcapt{#1}{#3}
%\vspace{+.5cm}
\end{figure}
}

\newcommand{\figpsautohb}[3]{
\begin{figure}[!h]
\begin{center}
\epsfig{file=#1.ps, width=#2cm}
\end{center}
\vspace{-.7cm}
\figurcapt{#1}{#3}
%\vspace{+.5cm}
\end{figure}
}

\newcommand{\figpsautop}[3]{
\begin{figure}[p]
\begin{center}
\epsfig{file=#1.ps, width=#2cm}
\end{center}
%\vspace{-.5cm}
\figurcapt{#1}{#3}
%\vspace{+.5cm}
\end{figure}
}

\newcommand{\figpsautobp}[3]{
\begin{figure}[bp]
\begin{center}
\epsfig{file=#1.ps, width=#2cm}
\end{center}
%\vspace{-.5cm}
\figurcapt{#1}{#3}
%\vspace{+.5cm}
\end{figure}
}

\newcommand{\figpsautotp}[3]{
\begin{figure}[tp]
\begin{center}
\epsfig{file=#1.ps, width=#2cm}
\end{center}
%\vspace{-.5cm}
\figurcapt{#1}{#3}
%\vspace{+.5cm}
\end{figure}
}

\newcommand{\figpsautot}[3]{
\begin{figure}[!tp]
\begin{center}
\epsfig{file=#1.ps, width=#2cm}
\end{center}
%\vspace{-1cm}
\figurcapt{#1}{#3}
%\vspace{+1cm}
\end{figure}
}

\newcommand{\figpsautob}[3]{
\begin{figure}[!bp]
\begin{center}
\epsfig{file=#1.ps, width=#2cm}
\end{center}
%\vspace{-1cm}
\figurcapt{#1}{#3}
%\vspace{+1cm}
\end{figure}
}


\newcommand{\ruleon}{
\begin{flushleft}
\rule{5.5cm}{0.1mm}\vspace{-0.1mm}
\end{flushleft}}
\newcommand{\ruleoff}{\rule{5.5cm}{0mm}\vspace{0mm}}

%%% Fortsaet tabel:

\newcommand{\ftabel}{
\renewcommand{\footnoterule}{\ruleoff}
\begin{table}[thbp]
\begin{minipage}[t]{\tabwidth}
\addtocounter{table}{-1}
\tabcapt
\small
\begin{sffamily}
\begin{center}
\begin{tabular}{\tabcols}
\hline
\hline
\tabhead
\hline
\hline
}

%%% Start tabel:

\newcommand{\btabel}[6][\textwidth]{
\providecommand{\tabhead}{tom}
\renewcommand{\tabhead}{#5\\}
\providecommand{\tabcols}{tom}
\renewcommand{\tabcols}{#3}
\providecommand{\tabtxt}{tom}
\renewcommand{\tabtxt}{#6}
\providecommand{\tabcapt}{tom}
\renewcommand{\tabcapt}{\tabelcapt{#2}{#4}}
\setlength{\tabwidth}{#1}
\addtocounter{table}{1}
\ftabel
}

%%% Afbryd tabel:

\newcommand{\atabel}{
\hline
\end{tabular}
\end{center}
\begin{flushright}
Forts\ae ttes ...
\end{flushright}
\end{sffamily}
\end{minipage}
\normalsize
\end{table}
\renewcommand{\footnoterule}{\ruleon}
\renewcommand{\tabcapt}{\caption{\tabtitle{fortsat}}}
}

%%% Afslut tabel:

\newcommand{\etabel}{
\hline
\hline
\end{tabular}
\end{center}
\end{sffamily}
\begin{footnotesize}
\tabtxt
\end{footnotesize}
\end{minipage}
\normalsize
\end{table}
\renewcommand{\tabtxt}{}
\renewcommand{\footnoterule}{\ruleon}
}

%%% Now from kbldef.sty found at ~mv
%%% xtra commands and caption definitions

\newcommand{\bce}{\begin{center}}
\newcommand{\ece}{\end{center}}
\newcommand{\bmath}{\begin{math}}
\newcommand{\emath}{\end{math}}
\newcommand{\bp}{\begin{minipage}[t]}
\newcommand{\ep}{\end{minipage}}
\newcommand{\hs}{\hspace*{.1in}}

%--------------------------------------------------------------------------
%FIGURE CAPTION
\newcommand{\textlineskipkbl}{\baselineskip=13pt}
\newcommand{\smalllineskipkbl}{\baselineskip=13pt}
\newcommand{\fcaptionkbl}[1]{
        \refstepcounter{figure}
        \setbox\@tempboxa = \hbox{Fig.~\thefigure. #1}
        \ifdim \wd\@tempboxa > 5in
           {\begin{center}
        \parbox{5in}{\smalllineskipkbl Fig.~\thefigure. #1 }
            \end{center}}
        \else
             {\begin{center}
             {Fig.~\thefigure. #1}
              \end{center}}
        \fi}
\newcommand{\figurecaptions}{   % hack loesning! 20/6 1993, KBL
        %\refstepcounter{figure}
        \setbox\@tempboxa = \hbox{\Large\bf Figure Captions}
        \ifdim \wd\@tempboxa > 5in
            {\parbox{5in}{\Large\bf Figure Captions}}
        \else
            {\Large\bf Figure Captions}
        \fi}
%--------------------------------------------------------------------------
%TABLE CAPTION
\renewcommand{\thetable}{\Roman{table}}
\newcommand{\tcaptionkbl}[1]{
        \refstepcounter{table}
        \setbox\@tempboxa = \hbox{Table~\thetable \vspace*{0.1cm}\\\noindent #1}
        \ifdim \wd\@tempboxa > 5in
           {\begin{center}
        \parbox{5in}{\smalllineskipkbl Table~\thetable \vspace*{0.1cm}\\\noindent #1 }
            \end{center}}
        \else
             {\begin{center}
             {Table~\thetable \vspace*{0.1cm}\\\noindent #1}
              \end{center}}
        \fi}
%--------------------------------------------------------------------------

%%% End of what was from kbldef.sty found at ~mv

%%% Now from jh_phd.sty found at ~mv
%%% Needed to include papers as apdxs
%  Chapters
%
%\def\@makechapterhead#1{ \vspace*{2pc} {\centering
% \Huge \bf \thechapter \par
% \vspace{10pt} \Large\sf \uppercase{#1}\par
% \nobreak \vspace{20pt}} }
%
%\def\@makeschapterhead#1{ \vspace*{2pc} {\centering
% \Large\sf \uppercase{#1}\par
% \nobreak \vspace{20pt}} }
%
%
%  specification of \newfileline will cause a line break in any
%  table of contents file.  This allows long or complicated titles.
%
\let\newfileline\relax             % will give linebreaks in .aux
%\let\n@wfileline\relax             % will give linebreaks in .toc
%

%\newcommand{\captionkbl}[1]{
%        \refstepcounter{figure}
%        \setbox\@tempboxa = \hbox{\small \sc Figure~\thefigure. \sf #1}
%        \ifdim \wd\@tempboxa > 5in
%           {\begin{center}
%        \parbox{5in}{\smalllineskipkbl\small \sc Figure~\thefigure. \sf #1 }
%            \end{center}}
%        \else
%             {\begin{center}
%             {\small \sc Figure~\thefigure. \sf #1}
%              \end{center}}
%        \fi}
%\renewcommand{\caption}{\captionkbl}

%\def\tableofcontents{\chapter*{Table of Contents}
%   \pagestyle{plain}
%   \thispagestyle{empty}
%   %\setcounter{page}{-2}
%   {\let\footnotemark\relax  % in case one is in the title
%   \sf \@starttoc{toc}
%    }
%   \newpage}

\def\appendix{\par
 %\setcounter{chapter}{0}
 \setcounter{section}{0}
 %\def\thechapter{\Alph{chapter}}
 \cleardoublepage\thispagestyle{empty}
 \mbox{}\vspace{6cm}\bce{\Huge APPENDICES}\ece
 \newpage\mbox{}\thispagestyle{empty}}

\def\papers{\par
 %\setcounter{chapter}{0}
 \setcounter{section}{0}
 %\def\thechapter{\Roman{chapter}}
 \cleardoublepage\thispagestyle{empty}
 \mbox{}\vspace{6cm}\bce{\Huge PAPERS}\ece
 \newpage\mbox{}\thispagestyle{empty}}

%\renewcommand{\thefigure}{\thechapter.\arabic{figure}}
%\renewcommand{\thetable}{\thechapter.\arabic{table}}
\renewcommand{\thefigure}{\arabic{figure}}
\renewcommand{\thetable}{\arabic{table}}

%

\def\asect#1{
         \addtocounter{section}{+1}     % to get the heading number correct
         \markright{\Alph{section} \it #1} 
         \addtocounter{section}{-1}     % get back to the original value
         \section{#1} 
         \markright{\arabic{section} \it #1}}

%
\def\paper{\cleardoublepage  % Starts new right-hand page.
   \thispagestyle{empty}     % Page style of chapter page is 'empty'
   %\global\@topnum\z@        % Prevents figures from going at top of page.
   \@afterindenttrue         % Allows indent in first paragraph.  Change
   %\ifchapternotes\@startnotes\fi  % if chapter notes
   \let\\\relax                  % vers 1.3
   %\secdef\@paper\@schapter}   % to \@afterindentflase to omit indent.
   }
%

%%% End of what was from kbldef.sty found at ~mv



\newcommand{\ftn}[1]{{\footnotesize $^#1$}}

\newcommand{\stortaa}{\AA}
\newcommand{\ovs}[1]{\noindent\underline{\bf{#1}}}

\newcommand{\bd}[1][2]{\begin{list}{Fejl}{
\setlength{\leftmargin}{#1cm}
\addtolength{\leftmargin}{0.2cm}
\setlength{\labelwidth}{#1cm}}}

\newcommand{\ed}{\end{list}}
\newcommand{\emne}[1]{\item[\bf{#1}\hfill]}
\newcommand{\pkt}[1]{\item[#1\hfill]}

\newcommand{\bsd}{\begin{description}}
\newcommand{\esd}{\end{description}}

\newcommand{\bi}{\begin{itemize}}
\newcommand{\ei}{\end{itemize}}

\newcommand{\be}{\begin{enumerate}}
\newcommand{\ee}{\end{enumerate}}

\newcommand{\beq}{\begin{equation}}
\newcommand{\eeq}{\end{equation}}

\newcommand{\beqa}{\begin{eqnarray}}
\newcommand{\eeqa}{\end{eqnarray}}

\newcommand{\tekstfil}[2]{\subsubsection{#2}
\label{#2}
{\footnotesize 
%\verbatiminput{#1/#2}
}
}

\newcommand{\kildetekst}[2][source]{
\tekstfil{#1}{#2.pro}
}

\newcommand{\kildetekstp}[2][source]{
%\newpage
\tekstfil{#1}{#2.pro}
}

%\newcommand{\btext}{
%{\footnotesize
%\begin{verbatim}
%#1 
%\end{verbatim}
%} }
\newcommand{\tts}[1]{{\small {\tt #1}}}
\newcommand{\ttf}[1]{{\footnotesize {\tt #1}}}
\newcommand{\ttsc}[1]{\begin{center}{\small {\tt #1}}\end{center}}
\newcommand{\ttc}[1]{\begin{center}{\tt #1} \end{center}}
\newfont{\ttt}{cmtt8 scaled 1}

\newcommand{\margnote}[1]{\marginpar[\flushright \textsf{#1}]{\textsf{#1}}}
\newcommand{\marghead}[1]{\margnote{\bf #1}}

\newcommand{\side}[1]{page \pageref{#1}}
\newcommand{\ide}[1]{\texttt{!!! #1}}
\newcommand{\app}[1]{appendix \ref{#1}}
\newcommand{\defin}[1]{definition (\ref{#1})}
\newcommand{\kap}[1]{kapitel \ref{#1}}
\newcommand{\Kap}[1]{Kapitel \ref{#1}}
\newcommand{\chap}[1]{chapter \ref{#1}}
\newcommand{\Chap}[1]{Chapter \ref{#1}}
\newcommand{\afs}[1]{afsnit \ref{#1}}
\newcommand{\Afs}[1]{Afsnit \ref{#1}}
%\newcommand{\sect}[1]{section \ref{#1}}
\newcommand{\Sect}[1]{Section \ref{#1}}
\newcommand{\sectp}[1]{section \ref{#1}, \side{#1}}
\newcommand{\afsp}[1]{afsnit \ref{#1}, \side{#1}}
\newcommand{\fig}[1]{figure \ref{#1}}
\newcommand{\figp}[1]{figure \ref{#1}, \side{#1}}
\newcommand{\Fig}[1]{Figure \ref{#1}}
\newcommand{\lig}[1]{equation (\ref{#1})}
\newcommand{\Lig}[1]{Equation (\ref{#1})}
\newcommand{\tab}[1]{table \ref{#1}}
\newcommand{\Tab}[1]{Table \ref{#1}}
\newcommand{\refp}[1]{\ref{#1} page \pageref{#1}}
\newcommand{\C}[1][]{{\bf (CHECK #1)}}

\newcommand{\STR}{Str{\"{o}}mgren}
\newcommand{\strom}{Str\"omgrenfotometri}
\newcommand{\hip}{HIPPARCOS}
\newcommand{\bol}{b\o l\-ge\-l\ae ng\-de}

\newcommand{\alal}{$\alpha -\alpha$}
\newcommand{\BB}{Big Bang}
\newcommand{\BBN}{\BB\ kernesyntese}
\newcommand{\cai}{Ca \small{I}}
\newcommand{\chl}{$\chi_l$}
\newcommand{\chto}{$\chi^2$}
\newcommand{\chtor}{$\chi^2_{red}$}
\newcommand{\cf}{$c_F$}
\newcommand{\dl}{$\Delta \lambda$}
\newcommand{\enh}{$\Gamma_{enh}$}
\newcommand{\eps}[1]{log $\epsilon(#1)$}
\newcommand{\fe}{[$Fe/H$]}
\newcommand{\feh}{$[ \frac{Fe}{H} ]$ }

\newcommand{\fli}{$f(^6Li)$}
\newcommand{\gf}{\texttt{log} $gf$}
\newcommand{\rad}{\texttt{log} $\Gamma_{rad}$}
\newcommand{\gu}{$g_u$} 
\newcommand{\iso}{lithium isotopforholdet}
\newcommand{\lair}[1]{$\lambda_{air}$}
\newcommand{\lam}[1]{$\lambda$#1 \AA}
\newcommand{\lamlam}[2]{$\lambda\lambda$#1-#2 \AA}
\newcommand{\lgg}{log$g$}
\newcommand{\li}{lithium}
\newcommand{\lif}{lithiumforekomst}
\newcommand{\Li}{Lithium}
\newcommand{\Lif}{Li-forekomst}
\newcommand{\Lii}{Li {\small I}}
\newcommand{\litfit}{\textsf{LITFIT}}
\newcommand{\mic}{$\xi$}
\newcommand{\MV}{$M_V$}
\newcommand{\PM}{$\pm$}
\newcommand{\sn}{$S/N$}
\newcommand{\te}{$T_{eff}$}

\newcommand{\deut}{D}
\newcommand{\iiiHe}{$^3$He}
\newcommand{\ivHe}{$^4$He}
\newcommand{\iH}{$^1$H}
\newcommand{\niBe}{$^9$Be}
\newcommand{\Nvil}{$^6Li$}
\newcommand{\Nviil}{$^7Li$}
\newcommand{\vil}{$^6$Li}
\newcommand{\viil}{$^7$Li}

\newcommand{\cd}{CD -30$^0$18140}
\newcommand{\hda}{HD 140283}
\newcommand{\hdb}{HD 84937}
\newcommand{\hra}{HR 3578}
\newcommand{\hrb}{HR 4158}
\newcommand{\hrc}{HR 4533}
\newcommand{\hrd}{HR 784}
\newcommand{\acir}{{$\alpha$ Cir}}
%\newcommand{\HR1217}{HR 1217 }
\newcommand{\Ap}{Ap}
\newcommand{\Apstar}{Ap star}
\newcommand{\Apstars}{Ap stars}
\newcommand{\roApstar}{roAp star}
\newcommand{\roApstars}{roAp stars}
\newcommand{\roAp}{roAp}
\newcommand{\mmag}{\mathrm{mmag}}

\newcommand{\ds}{$\delta$ Scuti}
\newcommand{\dsv}{$\delta$ Scuti variable }

\newcommand{\aflev}{August 2000}

\newcommand{\araa}{Annu. Rev. Astron. Astrophys}
\newcommand{\PASP}{Publ. Astr. Soc. Pac.}
\newcommand{\pasp}{Publ. Astr. Soc. Pac.}
\newcommand{\pasj}{Publ. Astr. Soc. Japan}
\newcommand{\mnras}{MNRAS}
\newcommand{\aaps}{A\&AS}
\newcommand{\aap}{A\&A}
\newcommand{\aj}{AJ}
\newcommand{\apj}{ApJ}
\newcommand{\aapr}{A\&AR}
\newcommand{\apjl}{ApJL}
\newcommand{\apjs}{ApJS}
\newcommand{\caosp}{Contrib. Astron. Obs. Skalnate Pleso}

\newcommand{\aup}{a$_{upper}$}
\newcommand{\bup}{b$_{upper}$}
\newcommand{\alo}{a$_{lower}$}
\newcommand{\blo}{b$_{lower}$}
\newcommand{\xhigh}{$\mathbf{x}_{\mathit{high}}$}
\newcommand{\xlow}{$\mathbf{x}_{\mathit{low}}$}
\newcommand{\xhighm}{\mathbf{x}_{\mathit{high}}}
\newcommand{\xlowm}{\mathbf{x}_{\mathit{low}}}

\newcommand{\maup}{a_{upper}}
\newcommand{\mbup}{b_{upper}}
\newcommand{\malo}{a_{lower}}
\newcommand{\mblo}{b_{lower}}
\newcommand{\eg}{e.g.}
\newcommand{\addtoindex}[1]{#1 \index{#1}}

\newcommand{\lign}{ligning}
\newcommand{\feks}{f.eks.}
\newcommand{\vha}{vha. }
\newcommand{\pga}{pga. }
\newcommand{\dvs}{dvs. }

\newcommand{\halpha}{H$_{\alpha}$}
\newcommand{\hbeta}{H$_{\beta}$}
\newcommand{\hgamma}{H$_{\gamma}$}
\newcommand{\hdelta}{H$_{\delta}$}
\newcommand{\teff}{T_{\mathit{eff}}}

\newcommand{\til}{$\sim$}
\newcommand{\Cai}{Ca{\small I}}
\newcommand{\Caii}{Ca{\small II}}
\newcommand{\Fei}{Fe{\small I}}
\newcommand{\Feii}{Fe{\small II}}
\newcommand{\Feiii}{Fe{\small III}}
\newcommand{\Sii}{Si{\small I}}
\newcommand{\Siii}{Si{\small II}}
\newcommand{\Tii}{Ti{\small I}}
\newcommand{\Tiii}{Ti{\small II}}
\newcommand{\Nii}{Ni{\small I}}
\newcommand{\Niii}{Ni{\small II}}
\newcommand{\Eui}{Eu{\small I}}
\newcommand{\Euii}{Eu{\small II}}
\newcommand{\Coi}{Co{\small I}}
\newcommand{\Coii}{Co{\small II}}
\newcommand{\Cri}{Cr{\small I}}
\newcommand{\Crii}{Cr{\small II}}
\newcommand{\Ci}{C{\small I}}
\newcommand{\Cii}{C{\small II}}
\newcommand{\Baii}{Ba{\small II}}


\newcommand{\sCai}{Ca{\tiny I}}
\newcommand{\sCaii}{Ca{\tiny II}}
\newcommand{\sFei}{Fe{\tiny I}}
\newcommand{\sFeii}{Fe{\tiny II}}
\newcommand{\sFeiii}{Fe{\tiny III}}
\newcommand{\sSii}{Si{\tiny I}}
\newcommand{\sSiii}{Si{\tiny II}}
\newcommand{\sTii}{Ti{\tiny I}}
\newcommand{\sTiii}{Ti{\tiny II}}
\newcommand{\sCoi}{Co{\tiny I}}
\newcommand{\sCoii}{Co{\tiny II}}
\newcommand{\sCi}{C{\tiny I}}
\newcommand{\sCii}{C{\tiny II}}
\newcommand{\sBaii}{Ba{\tiny II}}

\newcommand{\ix}{{\bf \small Ix}}

\newcommand{\cosib}{\cos i \cos \beta }
\newcommand{\sinib}{\sin i \sin \beta }
\newcommand{\tanib}{\tan i \tan \beta }
\newcommand{\chisq}{$\chi ^2$}

\newcommand{\arcs}{^{\scriptscriptstyle {\prime\prime}}} % arcsec
\newcommand{\farcs}{^{\scriptscriptstyle {\prime\prime}}\! \! . \: \!} % fraction of arcsec
\newcommand{\farcm}{^{\scriptscriptstyle {\prime}} \! \! . \: \!} % fraction of arcmin
\newcommand{\degree}{^{\circ}} %degree
\def\deg{\hbox{$^\circ$}}
\def\la{\mathrel{\hbox{\rlap{\hbox{\lower4pt\hbox{$\sim$}}}\hbox{$<$}}}}
\def\ga{\mathrel{\hbox{\rlap{\hbox{\lower4pt\hbox{$\sim$}}}\hbox{$>$}}}}
\def\sq{\hbox{\rlap{$\sqcap$}$\sqcup$}}
\def\arcmin{\hbox{$^\prime$}}
\def\arcsec{\hbox{$^{\prime\prime}$}}

\def\spose#1{\hbox to 0pt{#1\hss}}
\def\simlt{\mathrel{\spose{\lower 3pt\hbox{$\mathchar"218$}}
     \raise 2.0pt\hbox{$\mathchar"13C$}}}
\def\simgt{\mathrel{\spose{\lower 3pt\hbox{$\mathchar"218$}}
     \raise 2.0pt\hbox{$\mathchar"13E$}}}




\makeindex
