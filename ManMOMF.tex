\documentclass[]{article}
\pagestyle{plain}
\setlength{\oddsidemargin}{0cm}
\setlength{\headheight}{0cm}
\setlength{\headsep}{0cm}
\setlength{\topmargin}{0cm}
\setlength{\textwidth}{15.5cm}
\setlength{\textheight}{22.0cm}
% \setlength{\footheight}{1cm}
\setlength{\footskip}{2cm}
\renewcommand{\baselinestretch}{1.1}
\begin{document}
\noindent {\em Multi Object Multi Frame photometric package (MOMF) 
- Institute of Astronomy}
\newline \rule{15.5cm}{0.3mm}
\vspace{0.1mm}
\newline {\bf MANUAL -- DESCRIPTION OF MOMF SOFTWARE}
\newline \rule{15.5cm}{0.3mm}
\newline 2001 - Version 3.4
\vspace{3cm}
\Huge
\newline {\bf MOMF 3.4}
\vspace{3cm}
\normalsize
\newline {\bf Hans Kjeldsen and S{\o}ren Frandsen}
\vspace{6mm}
\newline {\bf hans@obs.aau.dk,srf@obs.aau.dk}
\vspace{4mm}
\newline {\bf Institute of Physics and Astronomy}
\vspace{1mm}
\newline {\bf Aarhus University}
\vspace{1mm}
\newline {\bf Langelandsgade}
\vspace{1mm}
\newline {\bf DK-8000 Aarhus C}
\vspace{1mm}
\newline {\bf Denmark}
\vspace{5mm}
\newline {\bf Contributions from Torben Arentoft and Martin R. Knudsen are acknowledged}
\normalsize
\newpage
\section{The MOMF software}
MOMF was first developed in 1989--1991 as a small software package 
written in
FORTRAN on a UNIX computer. It has been modified several times.
This release adds the possibility to read FITS files directly
and a new FIND routine, which is faster and more complete, we hope.
The present version also include earlier modifications to
make it easier to use on a cluster of workstation. A few minor changes
has been included, but functionally there are no major changes.
The package is mainly built to solve 
the
specific problems related to time-resolved CCD photometry, and therefore
MOMF is designed for handling large numbers of CCD frames.
The scripts for time series photometry are run in a batch-like
mode,
because it takes quite a long time to run the photometry program
for e.g. 500 stars on 350 frames. The present version has been reorganized
so that it is prepared to run in parallel, using PVM, 
on a number of workstations or
on a parallel computer. The parallel version has not been produced yet, but
it is the plan to offer such a parallel implementation as well.
All the major parts of the reorganization of the code has been
done already, and it is a minor job to write the parallel version.
The different time series scripts take, as input, a
number of CCD frames which contains the photometry for one cluster/field.
The output from running all the scripts is light curves for the
stars in the cluster/field together with error/noise estimates for each star.
\section{MOMF reduction techniques}
In time series observations
the same area is observed for up to 7--8 hours throughout the
night, which results in about 500 CCD frames (depending on the
exposuretime) for each object/field 
per night.
It is not a simple problem to reduce such a large number of 
CCD frames, but inspired by earlier reductions (Frandsen \& Kjeldsen,
1988, 1990, Kjeldsen \& Frandsen, 1989) where the 
DAOPHOT package (Stetson, 1987)
was used, we have developed software to reduce this kind of data.
The software, called MOMF (Multi Object Multi Frame photometry), has been
installed on the computer systems in Aarhus (HP, SGI) as well as on the HP 
computer
at Nordic Optical Telescope on La Palma and on the mc7 Sun computer-system at
European Southern Observatory (ESO), Garching in Germany.
Data reductions have been made at all these institutions (see e.g. 
Kjeldsen and Frandsen, 1992 and Jones {\em et al.}, 1992).
As a part of the following paper we have described the techniques 
used
to optimize the CCD reductions by use of the MOMF photometric 
package,
\begin{itemize}
\item Kjeldsen, Hans \& Frandsen, S{\o}ren., 1992,
 ``High-precision time-resolved CCD photometry",
{\em Publ. of the Astron. Soc. of the Pacific},
{\bf 104}, 413
\end{itemize}
The following sections will give only a brief description of the
reduction techniques as details can be found in the paper
(Kjeldsen and Frandsen, 1992).
\subsection{High-precision time-resolved CCD photometry}
Before deciding what kind of procedures MOMF was to contain, we
tried to estimate the noise level in different algorithms. Later in the
process, we also used synthetic images to compare {\em input 
magnitudes}
and {\em output magnitudes} and finally we used
real images to compare theoretical
noise with observed noise. Apparently we have understood the main 
part of the noise.
There were, at least, two reasons for
making special software for time series reductions of CCD frames,
instead of using one of the many photometry packages, which have been available 
since the early eighties e.g. ROMAFOT (Buonanno {\em et al.} 1983) and
DAOPHOT (Stetson, 1987). First of all, none of these packages could
be directly used to construct light curves for many stars in a way we
wanted.
In addition both ROMAFOT and DAOPHOT are intended for crowded
field photometry (globular clusters), which could introduce some 
additional errors when working
in semicrowded and noncrowded fields, such as open clusters.
 
\section{Time series photometry}
Once we are through the preliminary steps (basic reductions: Bias, 
clean, flat
fields, cosmic rays etc.), we
can proceed to the specific jobs for time series
photometry. 
The reduction procedure is as follows
\begin{itemize}
\item One image is selected as a reference image. Then a
FIND routine is run on the
reference image and useful reference stars are selected 
on the frame.
\item Construction of a {\em sum image} containing the
sum of some or all images in the time series.
In order to correct for
the tracking errors, the pixel values are moved before the 
summation.
The offset is determined by use of the reference stars.
\item A new run of FIND is used on the
sum image to detect as many stars as possible. Synthetic images
have been used to check the FIND algorithm.
Useful PSF stars are selected on the sum image.
\item The main MOMF programmes are the scripts for 
making
the photometry. These scripts use the combined PSF/AP
method.
The output from these programmes is, apart from lists of
photometry for each frame, light curves for the individual stars.
\end{itemize}

It is not always a good idea to do the sum image. It takes time and
if some images are of poor quality the sum is no better than a
good image. An option has been included in the new version to
leave out the process of summing and redetermining the position
of objects and PSF stars for the time series. You just call another
script.
\subsection{The sky background}
An accurate determination of the sky background is extremely
important for accurate photometry and as noise calculations shows,
this is especially the case for the fainter stars (discussed by Howell, 
1989).
There are two main techniques to determine 
the value of the sky background,
global sky determination and local sky
determination.
 
The global sky determination is a global determination of the 
background,
where the sky value is described by a simple, slowly changing, 
function of
the position in the field, e.g. a tilted plane. This can be a
precise sky determination, but the problem is if the sky background
can be described by a simple function.
 
We talk about a
local sky determination, when the sky background is calculated 
from the pixels around the star. In a field where all the stars are
single stars and no crowding is present, the local determination gives 
precise
values (tested on synthetic images). However, some stars are not 
single
stars, and there might be faint stars near some of the stars, which
can destroy the local determination.
 
The sky background in the MOMF photometry scripts is a
local determination (with a possibility to select a global
determination), but we have tried to
remove all {\em known} objects from the
background before we calculated the value for the background.
We have tested
3 different methods to determine the background value after
removing known objects and
they seem to give almost the same result. 
 
\subsection{MOMF Photometry --- General Philosophy}
From calculations of the intrinsic noise in different photometric
methods,
it is clear that
for non- or half crowded photometry, PSF fitting is the most precise
method for faint stars and aperture photometry is the most
precise method (if no neighbouring stars are present) for brighter 
stars.
In order to benefit from these conclusions and try to avoid some of 
the
other sources of noise (e.g. neighbouring stars, bad star coordinates 
and
bad PSF) we intended to develop a general method which {\em 
combines}
the ideas of aperture photometry and PSF fitting in order to
get as low a noise as possible. As a general criteria the scripts had 
to
be fast and without too much
interactive input in order to be useful for time series calculations.



\subsection{MOMF Photometry --- Combined PSF/AP}
The combined aperture and profile fitting scripts (PSF/AP routine) 
made for MOMF work in the following way:
\paragraph{Point Spread Function.}
From the PSF stars found on the {\em sum image} the PSF is 
calculated and the result is stored as a two component model: (1) A 
description (non-Gaussian) of the stellar core, and (2) a table 
containing the residuals, used as corrections from the stellar core 
description. It is a fully empirical PSF.
\paragraph{Local sky background determination I.}
For each star the sky background is found, using the previously 
described
sky routine, but without corrections for known {\em fainter} 
neighbouring stars.
In this first estimate we only remove light from neighbouring stars, 
which
are brighter than the present star. This
sky value is only used for the profile fitting.
\paragraph{PSF photometry.}
Using the empirical PSF, profile fitting and PSF photometry is
done for each star on the frame. Starting with the brightest star on 
the frame,
we remove the stars from the frame, when the photometry is done
(using the profile fit). This routine will slowly
clean the image for detected stars, and the PSF photometry will, 
therefore, be
affected only by fainter stars than the present star.
\paragraph{Local sky background determination II.}
In the clean frame a second background determination is done, again
using the previous described sky routine. The sky value is now much 
more
precise and not affected by detected stars.
\paragraph{Aperture photometry corrections.}
In order to correct for errors introduced by errors in the PSF, we
finally correct the PSF magnitude by use of aperture photometry.
As the important errors normally occur near the centre of the star
where the signal is largest, we only have to use small apertures
in order to correct the magnitude. Only for bright stars
will large apertures improve the precision. We have used 6 different
apertures scaled with the frame seeing in order to select the {\em 
best}
aperture later in the reduction procedure.
 
\subsection{MOMF Photometry --- Error elimination}
\paragraph{The noise}
Because we are able to use different apertures, the noise in the
PSF/AP photometry will always be lower or equal to the noise
introduced by using PSF or Aperture photometry alone. For bright
stars, big apertures might give the best result, but for faint stars,
small apertures might be best.
\paragraph{Neighbouring stars}
Because we make the final aperture photometry in
the clean frame, errors introduced by neighbouring stars (if
not very crowded) will be corrected by the final aperture routine.
Compared to normal profile fitting, especially the faint stars will
also be less affected by neighbouring stars using MOMF.
\paragraph{Determination of sky background}
For obvious reasons we are in a better position to find the {\em
correct} value for the sky background, when working in the clean
image.
\paragraph{Bad centre coordinates and the PSF}
Again, due to the final aperture photometry corrections, we have
a chance to correct errors introduced by the profile fitting 
(e.g. fitting errors and PSF changes over the field) and
errors introduced by
bad coordinates for the star position. From the clean images 
it is apparent
how difficult it is to
calculate the correct light level all over the profile.

\subsection{The PSF model}
Before we began to describe the PSF model for MOMF, we studied
different PSF models used in crowded field photometry (Stetson 
1987,
Buonanno {\em et al.} 1983). The MOMF PSF model does not
differ in idea from the models described in these photometry 
packages,
however, we did not make a very restricted description of the stellar 
core.
 
The PSF is defined as a two component model: (1) a core description
and (2) a residual description. Apart from a scaling parameter 
($ADU_{0}$)
for the profile the two components are described as
\paragraph{PSF Core.}
If $r$ is the radius from the centre position and $S(r)$ is a
function (in case of a Gaussian core profile it is a constant), we
describe the PSF core as
\begin{equation}
ADU_{core} (r) = ADU_{0} \cdot \exp ( \frac {-r ^{2} }{S^{2} (r) } )
\end{equation}
\paragraph{PSF residuals.}
If we define $z = x - x_{0}$ and $w = y - y_{0}$, where $(x_{0} , 
y_{0})$
is the centre position, the PSF residuals are described as a
look up table containing additive corrections from the
PSF core and the actual observed PSF
\begin{equation}
ADU_{residual} (z,w) = ADU_{0} \cdot R(z,w)
\end{equation}
\paragraph{Total PSF}
The PSF is then described by $ADU_{0}$, $S(r)$,  $R(z,w)$ and the
centre position $(x_{0} , y_{0})$. The PSF in
each point is found as $ADU_{core} + ADU_{residual}$.

\subsection{Obtaining the PSF}
The PSF is obtained from observed stars, and the PSF is
described empirically in terms of $S(r)$ and $R(z,w)$.
We compute $S(r)$ and $R(z,w)$ for each PSF star, and the final
model contains {\em clean} mean values for these parameters.
$S(r)$ and $R(z,w)$ are found as follows.

For each PSF star we first estimate a value for $ADU_{0}$. After that
we calculate values for $S(r)$ using
\begin{equation}
S(r) =  r \cdot ( \ln ( ADU_{0} / ADU ) )^{- \frac{1}{2} } 
\end{equation}
for different $r$ values,
and finally we make the look up table of residuals
\begin{equation}
R(z,w) = ADU / ADU_{0} - \exp ( \frac {-r ^{2} }{S^{2} (r) } )
\end{equation}
 
\subsection{Output from MOMF}
The total output from MOMF comprises tables of raw photometry for 
each
star, tables of relative photometry (corrected relative to a
calculated mean magnitude for each frame) for each star, tables of 
colour/magnitude relative photometry (corrected relative
to stars with almost same colour and magnitude) and tables
for {\em total} and {\em internal} rms noise.
 
The {\em total} rms noise ($\sigma _{total}$) is simply $\sigma$ rms 
for
the relative photometry,
while the {\em internal} rms noise ($\sigma _{internal}$) is 
calculated 
from the difference between neighbouring points on the light curve.
$\sigma _{internal}$ is found as
\begin{equation}
\sigma ^{2} _{internal} =  \frac {\sum _{i=1} ^ {N-1} ( m_{i} - m_{i+1}
) ^{2} } {2(N-1)}
\end{equation}
where $N$ is the number of points on the light curve and $m_{i}$ 
and
$m_{i+1}$ are, respectively, point number $i$ and $i+1$ on the curve.
The {\em internal} noise tells about the intrinsic noise and can be
compared with the theoretical noise calculations.
If we compute the ratio between {\em total} and {\em internal} 
noise, it gives
a direct indication of stars which shows {\em a too high} noise e.g.
variable stars. We call this ratio (detection level) for the $d$ $value$
and define it as
\begin{equation}
d = \sigma _{total} /  \sigma _{internal}   
\end{equation}
The d value is used to provide a preliminary detection of variable
stars (it has been used on NGC 6134 as well as other clusters, see
Frandsen and Kjeldsen, 1992).
\section{The MOMF Environment / 2-Byte Integers}
This version of MOMF (3.*) contains no software for the basic reductions
(preprocessing) such as bias- and dark-corrections, flat fielding,
corrections for cosmic rays and nonlinearity-corrections. The MOMF
version 3.* contains only the special photometric reduction software,
for doing multi object photometry on a large number of frames. Therefore
preprocessing has to be done by use of local existing software (e.g.
IRAF, MIDAS, IDL etc.). MOMF version 3.* can only process frames stored
in 2-Byte Integer-format and having no individual headers or frames in FITS
format. It
requires a special routine for the transformation from other types of
frame-formats (e.g. MIDAS, IRAF) into this simple MOMF Environment.
Users of MOMF are recommended to remove bias-scan areas before the
frames are processed by MOMF. MOMF version 3.* is optimized to process
frames having a typical value for the stellar-profile-FWHM between 3 and
7 pixels. If the frames show a higher or lower value for the FWHM, the
user is recommended to resample the frames in order to reach a FWHM
within the given limits. This resampling has to conserve the local-flux
in the frames. All frames shall be stored in the image directory
defined in an environment vareiable and
we recommend to use simple names, such as im.001, im.002, im.003... im.117 or
image1, image2, image3... image56, A01, A02, A03 etc..
Information
on the frame-size must be stored in the FORMAT file.
\vspace{4mm}

If your images contain data values that has the 16'th bit set, i.e.
values greater than 32767, then you must either rescale them to 15
bit numbers. The FORMAT file contain a last parameter, which will
let MOMF do this for you. Zero means no rescaling, and the program
will check that this is correct in the sense that only a few
negative values occur -- negative sky values due to statistics.
If the absolute value of the parameter is set to 1, negative numbers are understood as
high values (unsigned integers) and the image is rescaled to fit into the
positive range. If the absolute value of the parameter is set to 2, negative numbers
are accepted as negative numbers, but the image is rescaled
to fit the positive 15 bit range.
Finally, if this parameter is negative, the images are assumed to be in
FITS format, if positive or zero the image is in raw 2-byte integer format.
\begin{itemize}
\item
scale = 1:~~~~~Raw image with 16 to 15 bit rescaling
\item
scale = -3:~~~~~FITS image, no rescaling
\end{itemize}
\vspace{10mm}
\Large

{\bf References}
\normalsize
\begin{itemize}
\item
Buonanno, R., Buscema, G., Corsi, C.E., Ferraro,
I., Iannicola, G.  1983,
{\em Astron. Astrophys.,} {\bf 126,} 278
\item
Frandsen, S{\o}ren \& Kjeldsen, Hans, 1988, in
{\em Proc. Symp. on Seismology of the Sun \&}
{\em Sun like Stars,}
Tenerife, 26--30 September 1988, ESASP 286, p. 575
\item
Frandsen, S{\o}ren \& Kjeldsen, Hans, 1990,
{\em Proc. Confrontation between Stellar}
{\em Pulsation and Evolution}
Bologna, 28--31 May 1990,
{\em Conference Series of the Astronomical Society of the Pacific.}
Vol. 11, 312--315.
\item
Frandsen, S{\o}ren and Kjeldsen, Hans, 1992, {\em Observational Constraints 
on
mode excitation in $\delta$ Scuti stars in open clusters}
To appear in: Proc. IAU Coll. 137, Inside the Stars.
{\em Conference Series of the Astronomical Society of the Pacific.}
\item
Howell, S. B., 1989, {\em Pub. Astron. Soc. Pacific.,} {\bf 101}, 616
\item
Jones, A., Kjeldsen, H., Frandsen, S.,
Christensen-Dalsgaard, J., Hjorth, J., Sodemann
M., Thomsen, B. \& Viskum, M. 1992, ``Observations of
$\delta$ Scuti stars from {\AA}rhus",
{\em Proc. GONG 1992},
Boulder, CO, 11--14 August, 1992, to be published in
{\em Conf. Ser. Astron. Soc. of the Pacific.} - 1992.
\item
Kjeldsen, Hans \& Frandsen, S{\o}ren, 1989,
{\em ESO Messenger} {\bf 57}, 48
\item
Kjeldsen, Hans \& Frandsen, S{\o}ren., 1992,
{\em Publ. of the Astron. Soc. of the Pacific},
{\bf 104}, 413
\item
Stetson, P. B. 1987, DAOPHOT user's manual, DAO \& {\em
Pub. Astron. Soc.}
{\em Pacific}, {\bf 99,} 191
\end{itemize}
\newpage
\section{The MOMF Manual}
We can now continue to give a description of how to run the MOMF 3.0
software. In the following we give the name of the scripts (routines)
available in
the present version of MOMF and describe how to use these scripts. The
scripts are given in the order they have to be run when an actual
reduction is taken place. Each routine contains the following
descriptions,
\begin{itemize}
\item {\bf Before the run}, which gives information on what shall be done
before the routine is used.
\item {\bf Note}, which gives warnings and additional information
\item {\bf Input}, which describe the input you have to give the routine
while it is running.
\item {\bf Output}, which describe the output from the program.
\item {\bf See also}, which gives the name of other important scripts, which
should be known to exist before the present routine is used.
\end{itemize}

In this version 3.0 a major reorganization of the scripts has been made.
There are now only three main scripts plus one alternative one, which
you call one after the other (MOMF01, MOMF02, MOMF03 and the alternative
script MOMF01.nosum). The main scripts call a large set of
smaller scripts and FORTRAN programs.

When you install MOMF a directory is created called 'stdfil'.
Here are models for all the files that you need and we recommend 
to copy this set to a working directory. The initial setup thus
consists in the following procedures:
\begin{enumerate}
\item Define three or four environment variables:
\begin{itemize}
\item {\bf MOMF\_PROG} The directory where you installed the software.
\item {\bf MOMF\_DATAIN} The directory where your raw images are.
\item {\bf MOMF\_DATAOUT} The directory where your results are stored.
\item {\bf MOMF\_GRAD} If defined the new FIND routine is used.
\end{itemize}
\item Create a working directory and copy all files from the 'stdfil'
directory to your working directory. Some of the results will appear
in this directory, but most, including light curves, will be written
in the output directory defined by the variable. The main program scripts
are among the files copied.
\item Edit the files in the working directory:
\begin{itemize}
\item {\bf FORMAT} Defines the frame size and format
\item {\bf APrad} Defines the apertures used by MOMF and the maximum offset of a frame
\item {\bf PSFparam} Defines the PSF parameters to be used
\item {\bf SKYparam} Defines the sky parameters
\item {\bf input.SUM} The list of images for locating objects
\item {\bf input.MOMF} The full list of frames in the time series
\end{itemize}
\end{enumerate}
Often the only changes to be made are in FORMAT and the two 'input'
files.

\newpage
\noindent {\em Multi Object Multi Frame photometric package (MOMF) 
- Institute of Astronomy}
\newline \rule{15.5cm}{0.3mm}
\vspace{0.1mm}
\newline {\bf MOMF01 (MOMF01.nosum)}
\newline \rule{15.5cm}{0.3mm}
\newline {\em Run the FIND routine on the reference frame - selection of
reference stars - Make a sum image and repeat the process on this image}
\vspace{4mm}
\newline {\bf Before the run}
\begin{itemize}
\item Check that all frames is stored in the required format (see
description in section 5 of this Manual).
\item Check that frame-size information is stored in the FORMAT file.
\item Check that the image rescaling parameter is defined (last line)
\item Check that you are in the working directory
\item Define the environment parameter MOMF\_GRAD, if you want the fast FIND.
\item Be sure that you have all required information needed as input
(see description below).
\item Use this routine before any other
\end{itemize}
\vspace{2mm}
{\bf Note}
\newline MOMF01.nosum will stop after the first FIND and generate the
same files needed later as does MOMF01. The new 'gradfind' routine 
(see http://astro.ifa.au.dk/$\sim$vp4/projekt) reads
four additional parameters from the file {\tt find.param}. These parameters
are {\tt sigma}, {\tt segments}, {\tt cosmic} and {\tt 
skyback}. {\tt sigma} sets the sensitivity level of a test to screen
against non-stellar objects, {\tt segment}
divides the image in subareas but the parameter is not used in this
version of MOMF, cosmic regulate the test for cosmic ray traces,
and {\tt skyback} defines a value for the sky level only used in the
screening for cosmic events. 
Set {\tt sigma} to below zero to disable the test and start off
with 1 if you want to see what happens, when applying the test.
To disable the test for cosmic events set {\tt cosmic = 0} and
to have a test try {\tt 0.3}. To choose {\tt skyback} you need
to inspect your image.
\vspace{4mm}
\newline {\bf Input}
\begin{itemize}
\item Give the approx. seeing - FWHM - in pixels (e.g. 4.3).
\item Give a detection-level (sigma-like detections) (e.g. 8).
\item Then the routine will start to detect stars on the reference
frame. Some information is given on the screen. Of main interest is
column no. 3 (number of detected stars), column no. 6 and 7 (the frame
coordinates), column no. 8 (relative magnitude) and column no. 10
(approx. skybackground in the distance: 10-15 pixels from the star).
\item After the detection of stars, you will be given a possibility to
edit the coordinate file, by use of the vi-editor (Type [RETURN]).
leave the vi-editor by typing {\em :wq}
\item Then the program will ask for a saturation-limit in ADUs.
\item Then you must choose the number of PSF stars you want ($\leq 10$).
\item After this, the selection of reference stars for
XY-offset calculations and for defining the ensemble average will
be done. The program will show a number of stars which might be useful,
what you have to do is typing 1 (=yes) og 0 (=no) to the questions.
Or if you want to stop choosing stars, type 2 (=stop).
Up to 10 stars will be selected for the purpose.
At this point you should notice, whether the width of the stellar
images fill out the area displayed. In case they do, you should
consider starting all over after rebinning the images to reduce
the size of the stellar images (in pixels)
\item After the selection of suitable reference stars, you will be given
a possibility to enter a so-called Bad-pixel image (containing 1 in the
good pixels and 0 in the bad). If you have such an image/frame type 1
(=yes) and give the name og the frame, otherwise type 0 (=no).
\item The MOMF01.nosum script will stop at this point and you are ready
to continue with the next script MOMF02.
\item In determining the offsets of the images to be part of the
sum image you need to type how many iterations to go through.
The program gives for each image some information about the coordinates
of the PSF stars. And the mean of all stars to be used to calculate the offset.
\item The search for objects and PSF stars then continues as before
and the same information needs to be given like detection level etc.
\item Finally you are asked for the maximum offset to be permitted,
the saturation level and the script finishes with the PSF star
selection as before.
\end{itemize}
\vspace{2mm}
{\bf Output}
\vspace{4mm}
\newline The result of running MOMF01 is a set of files, where you might
be interested in inspecting a few of them.
{\sl coor.SUM} contains information about the stars in the frame.
{\sl PSFstars} give the list of PSF stars. {\sl im.SUM} is the raw sum image.
The last file does not exist of course if you used the script
MOMF01.nosum.
\newpage
\noindent {\em Multi Object Multi Frame photometric package (MOMF) 
- Institute of Astronomy}
\newline \rule{15.5cm}{0.3mm}
\vspace{0.1mm}
\newline {\bf MOMF02}
\newline \rule{15.5cm}{0.3mm}
\newline {\em The main photometric routine - PSF/AP photometry (MOMF)}
\vspace{4mm}
\newline {\bf Before the run}
\begin{itemize}
\item Check that you are in the working directory
\item Use this routine after MOMF01 or MOMF01.nosum
\item If you are rerunning MOMF02 remove the output files from
the output directory (defined by \$MOMF\_DATAOUT)
\item Check and change if you like
the input in the files {\em APrad, SKYparam} and {\em
PSFparam}. (see description below).
\end{itemize}
\vspace{2mm}
{\bf Note}
\newline MOMF02 will generate the photometric output from the
MOMF-run in the output directory as defined  by the variable
MOMF\_DATAOUT. Main output are the files \{frame\}.
The files {\em APrad, SKYparam} and {\em
PSFparam} contains the main input for the MOMF scripts.
The list of frames to be processed are in the file {\em input.MOMF}.
\vspace{4mm}
\newline {\bf Input}
\begin{itemize}
\item The MOMF02 needs input from the files {\em APrad, SKYparam} and {\em
PSFparam} as well as some of the output files from MOMF01.
\item APrad contains the following information: Radius of
aperture-radius used by the MOMF-aperture-corrections. These scale with
the FWHM. The file shall contain 6 increasing numbers. Optional one can define
the size of the search area for the displacement of frames relative to
the reference image. If the seventh number is missing the program 
defaults to 40.
\item SKYparam contains the following information: Radius of inner-
and outer-limit for the sky area. The sky-multiplication value can be
used to select a global sky-value, if the sky-background is found on
global basis and subtracted from the frame. Then a sky-multiplication of 0.0
will take 0 as the sky-level.
\item PSFparam contains the following information: Radius of
PSF-outer-limit in pixels. If this radius is larger than 26 pix. the PSF
will contain two residual tables - a small and a large - containing
residual description of inner and outer part of the PSF. A large radius
will give longer CPU-time per star. The PSF-residual can be weighted by
use of the PSF-core description in a given exponent (power of PSF). If
the second line in the file contains a 1 (=yes) this core control is in
action with an exponent given in the next line. The result of the
{\em image-clean}
procedure can be stored in a file empty.frame (however this takes
extra time and all frames will be given this name - deleting the
previus empty.frame).
\item 
The program is fully automatic. When running, the program gives
you some information on the screen. First it gives information on the
PSF-calculation. Then information on the PSF photometry, the combined
PSF/AP photometry and final information on
the output stored in the \{name\} files.
\end{itemize}
\vspace{2mm}
{\bf Output}
\vspace{4mm}
\newline The result of running MOMF02 is a number of files in the
output directory (see MOMF\_DATAOUT) containing
the photometry. Photometry for each frame is given in the files
\{frame\}. The first line in these files gives the
apertures used for the MOMF aperture-corrections. 
The second line gives for each aperture the zero point magnitudes
for correction to flux in ADU's:
\begin{equation}
-2.5*\log_{10}(flux(ADU)) = magnitude - zero point
\end{equation}
where the $zero point$ is from the second line and the magnitude from
line 3 and on (see below). The $zero point$ is defined as
\begin{equation}
zero point = 2.5*\log_{10}( \Sigma F(PSF)_i)
\end{equation}
where $F(PSF)_i$ is the flux in ADU's of the PSF stars.

From line 3 and on the
photometry for each star is shown (number, X, Y, magnitude 1 to 6).
In the file {\sl basic.PHOT} and {\sl offset.PHOT} additional 
information on each
frame is shown. The file {\sl basic.PHOT} gives XY-offset values,
skybackground in ADU/pix and FWHM-seeing in pix. The file {\sl offset.PHOT}
gives the XY-offset, the number of stars and the name of the file
containing the photometry for each frame. 
\newpage
\noindent {\em Multi Object Multi Frame photometric package (MOMF) 
- Institute of Astronomy}
\newline \rule{15.5cm}{0.3mm}
\vspace{0.1mm}
\newline {\bf MOMF03}
\newline \rule{15.5cm}{0.3mm}
\newline {\em This is post-processing program}
\vspace{4mm}
\newline {\bf Before the run}
\begin{itemize}
\item Check that you are in the working directory
\item Use this routine after MOMF02
\item If you rerun the routine remove the directories ABS and REL
in the output directory defined by \$MOMF\_DATAOUT
\end{itemize}
\vspace{2mm}
{\bf Note}
\newline MOMF03 will sort the photometric output from the
MOMF02 run in the output directory as defined  by the variable
MOMF\_DATAOUT. Main output are 
the files ABS/STAR.\#\#\# and
REL/STAR.\#\#\#. 
\vspace{4mm}
\newline {\bf Input}
\begin{itemize}
\item The MOMF03 needs input from the files produced by MOMF02 for 
each frame in the output directory.
\item In addition a file is needed with the name {\it MOMF.ref} in your
working directory, which defines the way the ensemble of reference
stars are defined, and the way weights defining the ensemble average
are calculated. First line contains an integer flag, where the choices
are -1: list of stars that follow should be excluded from the ensemble,
0: include all stars, 1: include stars in the list. The following lines
give the numbers of stars to be excluded/included. If the flag is
zero the lines following the first line are ignored. The numbers 
for the stars can
be chosen from the file {\it coor.SUM}.
The weights calculated by MOMF03 are based on the scatter about the
mean derived for each star. When all stars are included in the ensemble
average, the weights emphasize the bright stars below the normal
statitistics. If a smaller set
of stars are selected, the weights increase the importance of the
fainter stars above the normal statistics.
For rich fields with hundred or more stars, we recommend to
choose
the full ensemble average. For smaller ensembles the variable(s)
might have a disturbing influence and should be excluded from the
set of reference stars.
\end{itemize}
\vspace{2mm}
{\bf Output}
\vspace{4mm}
\newline The result of running MOMF03 is a number of files containing
information about individual stars.
containing the photometry for each frame. In the files ABS/STAR.\#\#\#
and REL/STAR.\#\#\#
photometry for each individual star can be found (ABS/STAR.\#\#\# -
raw magnitudes - and REL/STAR.\#\#\# - relative magnitudes). The files
SCAT.\#\#\# contains values for {\em magnitude minus mean divided by the
internal scatter}. The files REL/scatter... contains information on the
noise and error calculations, in the format: number, best aperture,
magnitude, number of frames used to calculate the rms-scatter, external
scatter (rms), number of points used to calculate the point-to-point
scatter, the point-to-point scatter, and the d-value.
\newpage
\end{document}
